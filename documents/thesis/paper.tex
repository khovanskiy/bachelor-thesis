% -*-coding: utf-8-*-
% This is an AMS-LaTeX v. 1.2 File.

\documentclass{report}

%\usepackage{pscyr}
%\renewcommand{\rmdefault}{fjn}
%\renewcommand{\ttdefault}{fcr}

%\usepackage{showkeys}
\usepackage[T2A]{fontenc}
\usepackage[utf8x]{inputenc}
\usepackage[english,russian]{babel}
\usepackage{expdlist}
\usepackage[pdftex]{graphicx}
\usepackage{amsmath}
\usepackage{pbox}
\usepackage{amssymb}
\usepackage{amsthm}
\usepackage{array}
\usepackage{amsfonts}
\usepackage{xspace}
\usepackage{algorithm}
\usepackage{algorithmicx}
\usepackage{amsxtra} 
\usepackage{sty/dbl12}
\usepackage{srcltx}
\usepackage{endnotes}
\usepackage{fnpct}
\usepackage{epsfig}
\usepackage{varwidth}
\usepackage{verbatim}
\usepackage{sty/rac}
\usepackage{algpseudocode}
%\usepackage[russian]{sty/ralg}
\usepackage{listings}
\usepackage{placeins}
%\usepackage{caption}
%\usepackage{floatrow}
\usepackage{caption}
\makeatletter
\captionsetup[table]{position=top,justification=raggedright,slc=off}

\def\BState{\State\hskip-\ALG@thistlm}
\makeatother
%\usepackage[
%    top    = 2.00cm,
%    bottom = 2.00cm,
%    left   = 3.00cm,
%    right  = 1.50cm]{geometry}
\hoffset = -10mm
\voffset = -20mm
\textheight = 230mm
\textwidth = 165mm

%%%%%%%%%%%%%%%%%%%%%%%%%%%%%%%%%%%%%%%%%%%%%%%%%%%%%%%%%%%%%%%%%%%%%%%%%%%%%%

% Redefine margins and other page formatting

%\setlength{\oddsidemargin}{0.5in}

% Various theorem environments. All of the following have the same numbering
% system as theorem.

\theoremstyle{plain}
\newtheorem{theorem}{Теорема}
\newtheorem{prop}[theorem]{Утверждение}
\newtheorem{corollary}[theorem]{Следствие}
\newtheorem{lemma}[theorem]{Лемма}
\newtheorem{question}[theorem]{Вопрос}
\newtheorem{conjecture}[theorem]{Гипотеза}
\newtheorem{assumption}[theorem]{Предположение}

\theoremstyle{definition}
\newtheorem{definition}[theorem]{Определение}
\newtheorem{notation}[theorem]{Обозначение}
\newtheorem{condition}[theorem]{Условие}
\newtheorem{example}[theorem]{Пример}
\newtheorem{algo}[theorem]{Алгоритм}
%\newtheorem{introduction}[theorem]{Introduction}

\floatname{algorithm}{Алгоритм}

\algnewcommand\algorithmicand{\textbf{and}\xspace}
\algnewcommand\algorithmicor{\textbf{or}\xspace}
\algnewcommand\algorithmicnot{\textbf{not}\xspace}
\algnewcommand\algorithmictrue{\textbf{true}}
\algnewcommand\algorithmicfalse{\textbf{false}}
\algtext*{EndWhile} % Remove "end while" text
\algtext*{EndIf} % Remove "end if" text
\algtext*{EndFor} % Remove "end for" text
\algtext*{EndProcedure} % Remove "end for" text

\renewcommand{\proof}{\\\textbf{Доказательство.}~}

%\def\startprog{\begin{lstlisting}[language=Java,basicstyle=\normalsize\ttfamily]}

%\theoremstyle{remark}
%\newtheorem{remark}[theorem]{Remark}
%\include{header}
%%%%%%%%%%%%%%%%%%%%%%%%%%%%%%%%%%%%%%%%%%%%%%%%%%%%%%%%%%%%%%%%%%%%%%%%%%%%%%%

\numberwithin{theorem}{chapter}        % Numbers theorems "x.y" where x
% is the section number, y is the
% theorem number

%\renewcommand{\thetheorem}{\arabic{chapter}.\arabic{theorem}}

%\makeatletter                          % This sequence of commands will
%\let\c@equation\c@theorem              % incorporate equation numbering
%\makeatother                           % into the theorem numbering scheme

%\renewcommand{\theenumi}{(\roman{enumi})}

%%%%%%%%%%%%%%%%%%%%%%%%%%%%%%%%%%%%%%%%%%%%%%%%%%%%%%%%%%%%%%%%%%%%%%%%%%%%%%


%%%%%%%%%%%%%%%%%%%%%%%%%%%%%%%%%%%%%%%%%%%%%%%%%%%%%%%%%%%%%%%%%%%%%%%%%%%%%%%

%This command creates a box marked ``To Do'' around text.
%To use type \todo{  insert text here  }.

\newcommand{\todo}[1]{\vspace{5 mm}\par \noindent
	\marginpar{\textsc{ToDo}}
	\framebox{\begin{minipage}[c]{0.95 \textwidth}
			\tt #1 \end{minipage}}\vspace{5 mm}\par}

%%%%%%%%%%%%%%%%%%%%%%%%%%%%%%%%%%%%%%%%%%%%%%%%%%%%%%%%%%%%%%%%%%%%%%%%%%%%%%%

\binoppenalty=10000
\relpenalty=10000

\begin{document}
	
	
	% Begin the front matter as required by Rackham dissertation guidelines
	
	\initializefrontsections
	
	\pagestyle{title}
	
	\begin{center}
		Санкт-Петербургский национальный исследовательский университет \\ информационных технологий, механики и оптики
		
		\vspace{2cm}
		
		Кафедра компьютерных технологий
		
		\vspace{3cm}
		
		{\Large Г. С. Ткаченко}
		
		\vspace{2cm}
		
		\vbox{\LARGE\bfseries
			Параллельные алгоритмы поиска \\ кратчайшего пути в графе}
		
		\vspace{4cm}
		
		Бакалаврская работа 
		
		\vspace{1cm}
		
		{\Large Научный руководитель: Г. А. Корнеев}
		
		\vspace{5cm}
		
		Санкт-Петербург\\ 2015
	\end{center}
	
	\newpage
	
	\setcounter{page}{4}
	\pagestyle{plain}
	
	%\dedicationpage{Put a dedication here}
	% Dedication page
	
	%\startacknowledgementspage
	% Acknowledgements page
	%{Put Acknowledgements here}
	
	% Table of contents, list of figures, etc.
	\tableofcontents
	%\listoffigures
	
	
	\def\t#1{\mbox{\texttt{\hbox{#1}}}}
	\def\b#1{\textbf{#1}}
	\def\tb#1{\t{\b{#1}}}
	
	\def\cln#1{\t{#1}}
	\def\pcn#1{\t{#1}}
	\newcommand{\p}{\par Здесь будет текст...}
	
	\def\drawfigure#1#2#3{
		\begin{figure}[ht]
			\centerline{ \includegraphics[width=8cm]{img/#1}}
			\caption{#2}
			\label{#3}
		\end{figure}
	}
	\def\drawfigurex#1#2#3#4{
		\begin{figure}[ht]
			\centerline{ \includegraphics[#4]{img/#1}}
			\caption{#2}
			\label{#3}
		\end{figure}
	}
	
	% Chapters
	\startthechapters
	%% Макрос для введения. Совместим со старым стилевиком.
\startprefacepage

	\input{chapters/chapter1.tex}
	\input{chapters/chapter2.tex}
	\input{chapters/chapter3.tex}
	\input{chapters/outro.tex}
	
	%\startappendices
	%\label{appendix}
	%\input{appendix}
	
	\bibliographystyle{sty/utf8gost705u}
	\bibliography{thesis}
	
\end{document}
