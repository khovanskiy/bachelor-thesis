\chapter{Детали реализации и тестирование}

\section{Работа с базой данных}
\subsection{Особенности базы данных}
Краткий принцип работы VoltDB, что такое in-memory и newSQL
\subsection{Персистентные модели данных}
Почему все модели транспортной сети персистентные (иммутабельные). Необходимость такого подхода.
\subsubsection{Транспорт}
Иерархия моделей. Разделение на модель и сущность.
\subsubsection{Остановки}
Модель остановки, страницы остановок, нормализация времени в период.
\subsubsection{Пересадки}
Как строим на основе страниц остановок.
\subsection{Кэши}
Зачем нужно? Географическая распределенность системы.
\subsubsection{Для моделей данных}
Время жизни объекта.
\subsubsection{Для запросов}
Хранение дерева сегментов по ключу запроса. Генерация ключей.
\section{Модуль для генерации карт транспортных рейсов}
\subsection{Генерация транспортных узлов}
Равномерная 2D генерация точек. Генерация признаков узлов. 
\subsection{Генерация транспортных рейсов}
Прокладка естественных маршрутов.
\subsection{Генерация центральных узлов}
Генерация точек-городов.
\section{Результаты тестирования}
Время запроса с 10с до 2с, прикрепить 3 графика. Сравнение со старой системой, сравнение с Йеном.
\chapterconclusion
Все сделано и работает. Скорость работы возросла, новые функции появились. Покрыто интеграционными и авто тестами.
