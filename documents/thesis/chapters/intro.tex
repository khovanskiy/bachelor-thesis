%% Макрос для введения. Совместим со старым стилевиком.
\startprefacepage
С увеличением объема грузовых и пассажирских перевозок растет и сложность обеспечения логистики, а также количество сервисов по продаже билетов в разные точки страны. 

К сожалению, существующие сервисы по поиску маршрутов обладают существенными ограничениями. Одни не строят мультимодальные маршруты, другие не обновляются в режиме реального времени. На момент написания статьи автор нигде не нашел сервисов, в которых бы строились удобные фильтры по доступных маршрутам. Вся фильтрация ограничивается только типом транспорта или типом вагонов, если система строит маршруты только для поездов.

При этом пассажироборот из года в год только растет. Например, только лишь пригородные перевозки в целом по сети РФ в 2011 году возросли и составили 878,33 млн чел.  А пассажирооборот пригородного железнодорожного транспорта по регионам России варьируется от 5\% до 30\% в общем пассажиропотоке. Не маловажный факт, что первое место по объему пригородных перевозок по итогам 2011 года со значительным отрывом занимает Московская железная дорога -- 510,1 млн чел., на которой тоже требуется эффективно строить маршруты.

Таким образом, целью данной работы является получение алгоритма, умеющего инкрементально строить маршруты, соответствующие входным условиям и упорядоченные по требуемой сортировке, а также доступные фильтры. На его основе требуется написать сервис, в котором будут проведены дополнительные технические оптимизации.
